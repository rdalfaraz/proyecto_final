\documentclass[a4paper,11pt]{article}
\usepackage[utf8]{inputenc} 
\usepackage[spanish]{babel} 
\usepackage{hyperref}
\usepackage{amsmath}
\usepackage{mathtools}
\everymath{\displaystyle}


\begin{document}
\title{Mi nuevo artículo} 
\author{Rodrigo Díaz Alfaraz}
\date{21 octubre 2018}
\maketitle

\paragraph*{Mi nuevo artículo}

\paragraph*{Autor}
Rodrigo Díaz Alfaraz

\section*{Resumen} 
Enlace al repositorio:
\url{https://github.com/rdalfaraz/proyecto_final} Es la url del repositorio donde se va a alojar este proyecto.
\paragraph*{Palabras clave}

GIT, LATEX, GITHUB, JABREF

\section*{Introducción}
Este articulo trata de recopilar los conocimientos adquiridos en el curso. 

\section*{Estado del arte}
En el ámbito de la investigación científica, el SoA (por sus siglas en inglés) hace referencia al estado último de la materia en términos de I+D, refiriéndose incluso al límite de conocimiento humano público sobre la materia.


\section*{Imágenes y tablas}
\section*{Fórmulas}
\paragraph{En esta sección se van a mostrar una serie de fórmulas:}
$$\cos (2\theta)= \cos^2 \theta - \sin^2 \theta$$
\linebreak
$$\lim\limits_{x \to \infty} \exp(-x) = 0$$

\section*{Bibliografía}

\end{document}
