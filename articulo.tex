\documentclass[a4paper,11pt]{article}
\usepackage[utf8]{inputenc} 
\usepackage[spanish]{babel} 
\usepackage{hyperref}
\usepackage{amsmath}
\usepackage{mathtools}
\everymath{\displaystyle}


\begin{document}
\title{Mi nuevo artículo} 
\author{Rodrigo Díaz Alfaraz}
\date{21 octubre 2018}
\maketitle

\paragraph*{Título}
Mi nuevo artículo

\paragraph*{Autor}
Rodrigo Díaz Alfaraz

\paragraph*{Resumen}  Enlace al repositorio:
\linebreak
\url{https://github.com/rdalfaraz/proyecto_final} \\Es la url del repositorio donde se va a alojar este proyecto.
\paragraph*{Palabras clave}

GIT, LATEX, GITHUB, JABREF

\paragraph*{Introducción}
Este articulo trata de recopilar los conocimientos adquiridos en el curso. 

\paragraph*{Estado del arte}
\paragraph*{Imágenes y tablas}
\paragraph*{Fórmulas}
En esta sección se van a mostrar una serie de fórmulas:
$\cos (2\theta) = \cos^2 \theta - \sin^2 \theta$
\linebreak
\linebreak
\linebreak
$\lim\limits_{x \to \infty} \exp(-x) = 0$

\paragraph*{Bibliografía}


\end{document}
